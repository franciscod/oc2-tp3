% Clase y configuracion de tipo de documento
\documentclass[10pt,a4paper,spanish]{article}
% Inclusion de paquetes
\usepackage{a4wide}
\usepackage{amsmath, amscd, amssymb, amsthm, latexsym}
\usepackage[spanish]{babel}
\usepackage[utf8]{inputenc}
\usepackage[width=15.5cm, left=3cm, top=2.5cm, height= 24.5cm]{geometry}
\usepackage{fancyhdr}
\pagestyle{fancyplain}
\usepackage{listings}
\usepackage{enumerate}
\usepackage{xspace}
\usepackage{longtable}
\usepackage{caratula}
\usepackage{hyperref}
\usepackage{graphicx}
\graphicspath{{img/}}

% Encabezado
\lhead{Organización del Computador II}
\rhead{Grupo: Fuga Villera Nro. 2}
% Pie de pagina
\renewcommand{\footrulewidth}{0.4pt}
\lfoot{Facultad de Ciencias Exactas y Naturales}
\rfoot{Universidad de Buenos Aires}

\begin{document}

% Datos de caratula
\materia{Organización del Computador II}
\titulo{Trabajo Práctico 3: Jauría}
%\subtitulo{}
\grupo{
	COMPLETAR
%	Grupo: Fuga Villera Nro. 2
%	\href{https://www.youtube.com/watch?v=wBNjDRXJNyY}{
%		\includegraphics[height=0.4cm,keepaspectratio]{YouTube-icon-full_color.png}
%	}
}

\integrante{Gabriel Matles}{397/12}{gabriel29m@gmail.com}
\integrante{Manuel Mena}{313/14}{manuelmena1993@gmail.com}
\integrante{Francisco Demartino}{348/14}{demartino.francisco@gmail.com}

\maketitle

\newpage

% Para crear un indice
%\tableofcontents

% Forzar salto de pagina
\clearpage

\section{Ejercicios}

\subsection{Ejercicio 1}

la la

lalal

ala
l
al
a
llal
a



%\section{Observaciones}
%	\includegraphics[keepaspectratio]{gen/diff.png}
%	\begin{enumerate}
%		\item un item
%		\item otro item
%	\end{enumerate}

% Otro salto de pagina
% \newpage

%\section{Resolución}


% \subsection{Ejercicio X}

%\subsection{Auxiliares}

\end{document}
